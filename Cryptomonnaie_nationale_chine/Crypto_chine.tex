

Cryptomonnaie nationale, l’expérience chinoise
De la crainte d’une concurrence privée à la souveraineté monétaire, à une souveraineté économique retrouvée grâce à une cryptomonnaie nationale, l’expérience chinoise.



De fin 2019 à mai 2022, les chinois ont effectué 83 milliards de yuans en transactions marchandes (12 milliards d’euros) dans la monnaie nationale digitale appelée e-CNY. C’est une sorte de cryptomonnaie d’État, gérée par la People’s Bank of China (PBOC), dont la valeur est associée à la devise nationale et garantie par l’État. Ce e-CNY n’est aujourd’hui encore qu’au stade de pilote, dans 23 villes et 15 provinces, et nécessitera des investissements importants pour adapter le fonctionnement et l’équipement des institutions financières, ainsi qu’une réforme importante des textes et régulations de la banque centrale. Le modèle économique de cette devise n’est pas encore défini, sa gestion et conversion sont gratuites pour tous les acteurs aujourd’hui, ce qui ne sera pas soutenable dans un modèle courant. Si elle est développée à usage de marché interne et du secteur de la vente aux particuliers dans un premier temps, il est possible d’envisager un usage plus international et stratégique d’une devise digitale dans un monde en fragmentation. 

L’initiative remonte à 2014, et a été formalisée en détail dans le plan Économie 2035 de 2020 qui définit la recherche dans ce domaine comme un axe stratégique : « faire progresser régulièrement la recherche et le développement de la monnaie numérique ». Ce programme national de la banque centrale est confié à M. MU Changchun, directeur de l’Institut de recherche pour la devise digitale. C’est déjà le deuxième plan quinquennal qui intègre le développement de cette devise digitale comme un axe stratégique, et coordonne les aspects de recherche, les questions d’investissements en infrastructure, les évolutions législatives, et la vision de l’économie pour la Chine. 



Le contexte de digitalisation du pays est très en avance sur les pays occidentaux. Le développement économique et la montée du pouvoir d’achat de la population se sont faits avec la digitalisation. Le développement du commerce a donc eu lieu, en parallèle avec les paiements digitaux. Depuis 20 ans, la majorité des transactions des particuliers s’effectue sur deux systèmes de paiement digitalisés sur smartphone : Alipay’s et WeChat pay. Les 83 milliards de yuans de transaction en trois ans en eRMB sont donc à rapprocher des 10 trillions de yuans mensuels en 2020 pour Alipay’s. 

L’enjeu du gouvernement Chinois dans l’acceptation par la population de cette nouvelle devise nationale n’est donc pas le passage au digital, mais le changement de support digital. Et cela passe, comme dans toutes les économies du monde, par la confiance puis par l’expérience utilisateur. La confiance en une cryptomonnaie d’État passe par la communication sur le caractère légal et garanti par l’État. La Chine n’emploie jamais le mot cryptomonnaie pour désigner le e-CNY, mais parle de yuan digital. 

La première interface utilisateur intuitive et pensée service arrive en janvier 2022 dans les 12 premières villes pilotes, juste avant l’essai à grande échelle pour les JO d’hiver à Pékin. Elle comprend les fonctions de paiement sur smartphone, mais également l’ensemble des transactions bancaires à distance, les virements et transferts d’argent, les échanges de e-CNY de particulier à particulier, et un porte-monnaie électronique e-CNY accessible en ligne et hors ligne. Tout cela sans frais dans la version pilote. 

Cette monnaie digitale doit également assurer les trois fonctions monétaires principales de stockage de valeur, unité de compte, et moyen d’échange. Les deux premiers étant garantis par l’État et la banque centrale, les prochaines étapes vers un pilote général se concentrent sur le moyen d’échange pour faciliter l’adoption par la population, et donc plus particulièrement sur le secteur de la distribution et vente aux particuliers. Un autre accélérateur pour l’adoption nationale de cette devise est le paiement des salaires, la population étant naturellement encline à dépenser son salaire dans la monnaie de paiement. Enfin, arrive en dernière étape, le paiement des taxes et services à l’État. Les évolutions législatives vont s’accélérer pour supporter l’adoption du e-CNY. 

La vision a été donnée par le directeur de l'Institut de recherche sur la devise digitale de la banque centrale de Chine, M. MU en juillet 2022, avec en priorité la réconciliation de deux objectifs antagonistes : la protection de la vie privée des utilisateurs de cette monnaie, et le respect des réglementations internationales de lutte contre le blanchiment de l’argent et contre le financement du terrorisme. Le concept retenu est celui de « l’anonymisation contrôlée », ne permettant un accès aux données de l’individu que suite à détection de flux financiers douteux. Les évolutions législatives concerneront également la certification d’opérateurs, les conditions de transferts de fonds, les sanctions en cas de transactions illégales. Une initiative nationale donc, mais qui pose des questions de stratégie monétaire et de gouvernance économique mondiale. Si la devise dominante actuelle dans le commerce international est le dollar américain, et le système transactionnel interbancaire SWIFT, les récents événements et l’isolement économique de la Russie posent la question de flux financiers alternatifs. La Russie a ainsi proposé à ses partenaires commerciaux un système alternatif au SWIFT pour les transactions bancaires. Une devise digitale garantie par l’État chinois pourrait tout aussi bien devenir demain une alternative au dollar américain pour les transactions commerciales internationales. La Chine a également un système de transaction interbancaire concurrent au SWIFT, le Cross Border Interbank Payment System (CIPS). C’est dans ce contexte de guerre Ukraine – Russie que l’ancien Gouverneur de la Banque Centrale M. ZHOU Xiaochaun a prononcé un discours au forum pour la finance globale à Tsinghua en avril 2022 se voulant rassurant sur le sujet : 

« Le yuan digital de la Chine est destiné aux transactions pour le commerce de détail en Chine, pour la commodité des gens ordinaires et petits commerçants, pas pour remplacer le dollar américain. (…) Mais il n’est pas exclu que le e-CNY ne puisse pas servir de paiement à l’international dans le futur, mais plus à des fins de commerce international. » 

Si la Chine est la première économie majeure à avoir lancé la digitalisation d’une devise nationale, elle n’est pas le seul pays à développer cette technologie. La première initiative a eu lieu aux Bahamas, dont la banque centrale avait lancée, en octobre 2020, le Sand Dollars. Selon un sondage de la Bank for International Settlements de fin 2019 auprès de 66 banques centrales majeures, 80 % d’entre elles déclaraient étudier la devise digitale face à la menace que représentait une potentielle devise privée (Facebook Libra). En 2020, peu de pays avaient un réel développement de devise digitale sous forme de pilote :

    Bahamas, avec une devise digitale lancée officiellement en 2020 ;

    Ukraine et Uruguay, avec des pilotes menés et finalisés en 2018 sans généralisation ;

    Caraïbes, Suède et Chine avec des pilotes lancés en 2020 et toujours en cours.

La Chine étant la seule grande puissance économique à développer à son échelle cette devise digitale, elle est également en position de définir de futurs standards mondiaux. Le Président de la République Populaire de Chine avait prononcé un discours au G20 le 21 novembre 2020 en ce sens où il avait appelé l’organisation « à discuter de l’élaboration des normes et des principes pour les monnaies numériques des banques centrales (CBDC) avec une attitude ouverte et accommodante, et à gérer correctement tous les types de risques et de défis tout en faisant pression collectivement pour le développement du système monétaire international ». Les BRICS, depuis le déclenchement de la guerre entre Russie et Ukraine, sont à la recherche de solutions alternatives économiques et financières. L’Argentine, et plus récemment l’Algérie, ont officiellement candidaté pour rejoindre l’organisation. De nombreux autres pays du Sud cherchent à s’en rapprocher. Même si la création du e-CNY avait pour but premier le marché intérieur chinois, les tensions mondiales et l’imposition des sanctions économiques et financières dans un conflit armé accélèrent la recherche de solutions alternatives pour le commerce mondial, et par conséquent l’intérêt pour le e-CNY. 

Les freins principaux restant à l’adoption internationale d’une devise numérique sont la capacité technologique et la définition des standards internationaux. Des contraintes qui trouveront une solution dans le temps tant que le financement sera assuré et la vision long terme maintenue. Les deux facteurs clés du succès de cette démarche chinoise sont le financement et la vision long-terme associés à une ressources humaine formée et disponible en grand nombre. Une question se pose sur la capacité d’autres États à mobiliser ces mêmes facteurs clés pour permettre la mise en œuvre d’une e-devise alternative sur le marché financier international, et ainsi offrir aux autres pays le choix entre plusieurs modèles économiques et sociétaux.
