\documentclass[a4paper]{article}

\usepackage[utf8]{inputenc}   
\usepackage[T1]{fontenc}      
\usepackage{geometry}  
\usepackage{hyperref}       
\usepackage[francais]{babel}  
                          

\title{TOP 5 des livres cyber pour les débutants !}          
\author{Contriburion M82} %\and Autre Auteur}
\date{21 juin 2023}                     
			    
\sloppy  
\begin{document}

\maketitle                 

La sélection M82

Chiffrement, red team, CERT, ISO27001, ANSSI… Le vocabulaire du cyber peut sembler hermétique pour un néophyte. Mais si ce domaine est vaste et technique, il reste cependant accessible à tous ceux qui souhaitent s’y intéresser ; et notamment ceux n’ayant pas de formation en informatique. 

L’objectif de cet article est de vous proposer un top 5 des ouvrages accessibles pour les débutants. Si ces livres ne feront pas de vous des experts, ils vous permettront de découvrir la grande variété des sujets appartenant au domaine du « cyber ». Libre à vous ensuite d’approfondir grâce à des ouvrages plus spécifiques.

Bonne lecture !

\begin{itemize}

\item \textbf{Cyberattaques de Gérôme Billois}, 2022, éd. Hachette, 239 p.
\\
La qualité d’un consultant réside souvent dans sa capacité à expliquer des choses complexes de manière accessible. C’est justement ce qu’arrive à faire Gérôme Billois avec cet ouvrage complet et au design soigné. Vous y trouverez des récits sur les différents types de cyberattaques, des explications sur les enjeux de la cyber sécurité pour les états, les entreprises et les individus ainsi que des portraits de professionnels représentant la grande diversité des métiers. Ce libre est le B.A.-BA pour tout nouvel entrant dans le domaine de la cybersécurité et notamment ceux qui se destinent à une carrière dans le conseil.



\item \textbf{Cyberstructure – L’Internet, un espace politique} de Stéphane Bortzmeyer, 2018, éd. C\&F éditions, 268 p.
\\
Écrit par un expert des questions d’internet et travaillant pour l’AFNIC (organisme gestionnaire du registre des noms de domaine en .fr), ce livre se divise en deux parties. La première se concentre sur le fonctionnement d’internet : les protocoles, les applications web, les organismes de gouvernance et quelques sujets particuliers tels que les crypto monnaies. La deuxième partie est dédiée à une réflexion sur les aspects politiques d’internet : doit-on limiter le chiffrement ? Les concepts de sécurité et vie privée sont-ils opposés ? Quelles technologies choisir pour l’internet de demain ?


\item \textbf{Sécurité et espionnage informatique – Connaissance de la menace APT} de Cédric Pernet, 2015, éd. Eyrolles, 220 p.
\\
Maitriser sa sécurité signifie que l’on doit comprendre les menaces auxquelles il faut faire face et c’est justement l’objet de ce livre consacré aux APT. Les APT, pour Advanced Persistent Threats, sont les menaces du haut du spectre (étatiques ou criminelles). L’auteur va ainsi définir ce terme qui fait débat au sein de la communauté cyber avant d’en décrire les différentes phases. L’exposé est enrichi d’exemples concrets et aborde certains aspects techniques. Il ne couvre cependant pas les actions à mener pour répondre à ce type de menace (la réponse à incident). Pour les débutants, il est intéressant de noter que si les APT sont des menaces avec un impact potentiel élevé, il ne s’agit pas forcément des attaques les plus sophistiquées techniquement.


\item \textbf{Les bases du hacking} de Patrick Engebretson, 2017, éd. Pearson, 220 p.
\\
Vous souhaitez concrètement comprendre comment se déroule une cyberattaque ? Alors ce guide est fait pour vous ! Patrick Engebretson, professeur américain en sécurité informatique, va vous présenter chaque étape d’une attaque (reconnaissance, scan, exploitation, maintien d’accès) à l’aide d’un cas d’étude. Grâce à ses conseils, aux outils gratuits et aux lignes de commande présentées vous serez en mesure de refaire l’attaque depuis votre ordinateur personnel. Bien que ce livre comporte des éléments techniques, il reste accessible dans sa grande majorité à tous.


\item \textbf{La Cyberdéfense – Politique de l’espace numérique} sous la dir. Amaël Cattaruzza, Didier Danet \& Stéphane Taillat, 2019, éd. Armand Colin, 255 p.
\\
Cet ouvrage collectif analyse la cyberdéfense sous le prisme des relations internationales. Les auteurs présentent l’état des connaissances scientifiques dans des sujets variés tel que le concept de guerre cyber, les enjeux de souveraineté numérique, le positionnement des grands pays cyber (États-Unis, Chine, Russie), le droit international appliqué au numérique… Plus théorique que pratique, cet ouvrage est un indispensable pour les étudiants ayant un projet de mémoire en lien avec la cyberdéfense.

\end{itemize}

Et voici deux coups de cœurs personnels que je recommande :
\\
\begin{itemize}

\item \textbf{Mémoires vives d’Edward Snowden}, 2019, éd. Seuil, 384 p.
\\
Après avoir travaillé pour la CIA et la NSA, Edward Snowden est devenu célèbre en diffusant auprès de journalistes occidentaux de très nombreux documents classifiés issus des agences de renseignements américaines et du ministère de la défense. Dans ce livre, le lanceur d’alerte relate sa carrière et les réflexions qui l’ont poussé à agir de la sorte. Il relate également ses activités de cyber espionnage et les capacités américaines en matière de renseignement technique. Cette histoire a été adaptée par Oliver Stone dans le film Snowden avec Joseph Gordon Lewits. Le documentaire Citizenfour de Laura Poitras relate également la relation entre E. Snowden et les journalistes ayant reçu les documents classifiés. 
\\
\item \textbf{25 énigmes ludiques pour s’initier à la cryptographie} de Pascal Lafourcade & Malika More, 2021, éd. Dunod, 209 p.
\\
Écrit par deux chercheurs de l’IUT d’informatique de l’Université Clermont Auvergne, ce livre a pour objectif de vous initier à la cryptographie, l’art de rendre illisible un message pour le protéger des regards indiscrets. À travers ces 25 énigmes, aux niveaux de difficulté variés, vous découvrirez des concepts de chiffrements dont certains sont toujours utilisés aujourd’hui. Les énigmes sont également agrémentées d’encadrés sur l’histoire du chiffrement. Rassurez-vous, pas besoin d’être un mathématicien chevronné pour résoudre ces énigmes. Un niveau minimum de lycée est conseillé, mais des indices vous aideront en cas de blocage. Bonne chance ! Si le sujet vous intéresse, je vous conseille le podcast français NoLimitSecu n°332 dédié au livre, avec Pascal Lafourcade en invité.
\\
\end{itemize}
Enfin je ne peux que vous recommander de suivre \href{https://secnumacademie.gouv.fr/}{le MOOC gratuit de l’ANSSI} (Agence nationale de la sécurité des systèmes d'information) qui est une très bonne introduction à la cybersécurité. Vous y apprendrez notamment les bonnes pratiques d’hygiène numérique qui s’appliquent aux personnes comme aux entreprises. 
\end{document}
