
Par Nicolas Chevrier
La lecture ou la relecture de l’ouvrage de Clarke et Knake, bien nommé Cyber War est encore très riche d’enseignement malgré une publication un peu datée, en 2010 (il y a eu une nouvelle édition par Harper Collins en 2012).


Douze années peuvent sembler bien longues dans un domaine aussi dynamique et évolutif que la cyber. Et pourtant, chaque étape de cet ouvrage a relativement peu souffert du temps écoulé. Il est probable que la saveur particulière des livres anglo-saxons y soit pour beaucoup. Un soupçon de pragmatisme et une construction bien ficelée : succession d’une histoire principale agrémenté d’anecdotes ; un peu de retour d’expérience dispensé ici et là, une vision et des propositions concrètes pour se projeter une fois la lecture achevée. Sans surprise, c’est exactement ce que l’on retrouve dans Cyber War.

Quelques carences sont toutefois présentes et autant commencer par celles-ci. De cette manière on sait ce que l’on ne trouvera pas en lisant cet ouvrage.

Du fait de sa publication en 2010, l’approche du sujet est très “réseau centrée” : au sens transport de l’information. Les deux auteurs font grand cas du rôle prépondérant que devraient jouer les opérateurs de télécommunication (les fameux “Tier 1”) dans la détection des cyber attaques qui sont perpétrées à l’encontre des États-Unis ou de leurs alliés. Un argumentaire relativement développé, visant à conférer aux dits opérateurs (1) l’obligation de surveiller le trafic, notamment entrant, (2) de l’analyser (le “Deep Packet Inspection” -DPI- laissait alors espérer un tas de choses) et (3) les moyens législatifs d’agir à l’encontre de tout trafic jugé malveillant. Ça fleure bon les années 2010 ! De plus, le développement plus récent des infrastructures de stockage de type clouds et le changement de paradigme sécurité du tout réseau vers des sondes data sont évidemment absents. D’ailleurs, l’approche technique est plutôt faible, mais ça n’est pas ce qui fait l’attrait de l’ouvrage. 

Les atouts de Cyber War sont tout autres et à vrai dire, plutôt nombreux. Le premier consiste en l’histoire contée de l’adoption du cyber par le Departement of Defense et les services de renseignement américains. On ne se pose plus vraiment la question aujourd’hui, mais les grands acteurs qui façonnent notre quotidien dans le cyberespace ont une histoire très récente. Encore plus que celle des armées de l’air créées majoritairement dans la première moitié du XXème siècle. Elle est d’ailleurs tout aussi mouvementée et passionnante pour qui s’intéresse plus aux capacités cyber qu’à la forme de l’empennage de tel ou tel aéronef ! Il s’agit donc de déterminer quel ancien “corps” sera le premier à conquérir ce nouvel espace, y attirer les crédits afférents et ainsi répondre aux défis futurs de la Nation comme de servir les intérêts de quelques ambitieux. 

Depuis des positions bien intégrées aux cercles politiques et décisionnels, Clarke (conseiller à Maison Blanche puis coordinateur de la cybersécurité) donne vie à cette aventure au début des années 90 alors que l’armée américaine s’interrogeait sur les possibilités de s’introduire au sein des systèmes de défense anti aérienne de l’armée irakienne pour appuyer les opérations militaires plus traditionnelles. Un dilemme toujours contemporain se fait alors jour, opposant l’entrave à la collecte de renseignement. L’entrave et, de manière générale, l’emploi de capacités cyber furent défendus très tôt par l’United State Air Force dont le directeur de la Task Force Cyber disait déjà en 2008 : “If you are defending in cyberspace, you’re already too late. If you do not dominate in cyberspace, then ou can not dominate in other domains.”

Toute ces déclarations n’étaient pas spécialement au goût des agences de renseignement et tout particulièrement de la National Security Agency qui préfère opérer en toute discrétion. La situation a finalement évolué vers la création d’un commandement dual-hatted, regroupant la NSA et le US Cyber Command sous une même autorité.

Les affaires militaires et du renseignement étant en quelque sorte entre de bonnes mains, les auteurs vont dès lors s’attacher à explorer l’épineux problème de la défense de la nation américaine, i.e. les infrastructures critiques, les entreprises, le gouvernement fédéral, etc. Clarke et Knake n’auront de cesse d’énumérer les nombreux renoncements de la politique américaine tant en termes de politique incitative d’intégration de la cybersécurité dans le développement du secteur technologique, l’édictions de standards de sécurité ou encore l’incarnation d’un leadership au sein de l’Etat fédéral. Si l’on sait que les Républicains ont toujours rechigné à mener une politique intrusive pour le secteur privé, les Démocrates n’ont pas sauté le pas non plus. Quant à la faiblesse de leadership, force est de reconnaître que les nombreux postes de coordinateur (que Clarke a notamment occupé) étaient essentiellement consultatifs, manquant de pouvoir de conviction et de coercition. De plus, ils étaient installés au sein d’un Department of Homeland Security (DHS) trop jeune et trop grand pour s’intéresser suffisamment aux défis et menaces issus du cyberespace. En effet, les années 2000-2010 étaient bien plus marquées par les conséquences de l’attaque terroriste du 11 septembre, i.e. la la lutte contre le terrorisme (le « War on Terror » de Georges W. Bush) et la conduite de deux guerres en Irak et en Afghanistan que par la menace probable d’un ensemble de geeks en “hoodies”, pianotant frénétiquement sur des claviers d’ordinateurs…

Notons qu’en 2010, il s’agissait d’un bilan audacieux et visionnaire en 2010, où la bascule de la lutte contre le terrorisme vers le concept de “Great Power Competition” ne s’était pas encore opérée. Les auteurs avaient notamment pressenti qu’en l’absence d’une défense à la hauteur des adversaires des États-Unis, il serait délicat d’employer l’arme cyber de manière offensive. 

Pour en arriver là, il conviendrait de développer une stratégie défensive initiale, sobrement baptisée “defensive triad” par les deux auteurs. Il s’agirait de mettre en oeuvre des critères de sécurité promulgués au travers de lois et réglementations fédérales. Celles-ci sont regroupées au sein de trois piliers :

•               La défense des opérateurs de transports de communications dits “Tier 1 operators” au travers desquels transite 90 % du trafic Internet nord américain. L’objectif serait ainsi de leur donner les moyens de détecter le trafic malveillant et les doter, grâce à un cadre réglementaire ad hoc, de l’autorité nécessaire au blocage du-dit trafic.

•               La sécurisation de la “power grid” américaine. Rappelons ici que l’alimentation électrique aux États-Unis ne fait pas l’objet d’un monopole comme en France et qu’une myriade d’opérateurs privés sont regroupés au sein de trois grandes “grilles”. Cela constitue tout à la fois un avantage, une forme de résilience par l’hétérogénéité des systèmes mais aussi une grande faiblesse. En effet, toute attaque réussie même avec un impact minime serait perçue comme une échec et un aveu de faiblesse de l’État américain (nda : et du Canada car les 3 grilles recouvrent toute l’Amérique du Nord.

•               Enfin, la défense du DoD… au travers des systèmes logiciels et matériels employés, la redondance de systèmes classifiés, l’intégration de la sécurité dans les grands programmes militaires de demain (les débuts du F35), etc.




Poursuivant la réflexion quant à l’emploi des capacités cyber, notamment offensives, les auteurs procèdent à une comparaison fort intéressante de la doctrine d’emploi de l’arme nucléaire dont a largement bénéficié (nda : certainement à tort) la doctrine cyber. Si vous avez toujours voulu savoir pourquoi l’on a parlé et l’on parle encore de dissuasion cyber (ou “cyber deterrence”) alors cette section vous éclairera. On comprend ainsi qu’avec une défense faible et un investissement important dans les capacités offensives, les États-Unis ont eu tendance à ériger une réalité qu’ils se sont imposés en doctrine, plutôt que de réfléchir aux objectifs stratégiques à atteindre. Dans ce cas là, un plan différent aurait certainement été adopté, travaillant à réduire leurs faiblesses pour se renforcer collectivement.

Enfin la stratégie américaine est mise à l’épreuve d’un exercice organisé au plus haut niveau de l’État. Cet exercice “tapis vert” ou “Table Top Exercice - TTX” pour reprendre la terminologie militaire voit s’affronter deux équipes, l’une chinoise et l’autre américaine, autour d’un scénario bien ficelé. Si l’on retrouve les bases d’un conflit traditionnel en mer de Chine, laissant craindre l’escalade dangereuse dans l’affrontement de bâtiments des marines des deux pays, l’emploi de capacités offensives cyber est rapidement placé au centre du scénario. Sans en révéler toute la teneur, l’équipe jouant les États-Unis choisit à un moment de mener des cyber attaques contre les infrastructures critiques civiles chinoises. L’objectif est d’envoyer un message fort, espérant ainsi forcer l’adversaire à reculer, sans engager un affrontement maritime incertain… Mais l’équipe chinoise ayant anticipé cette possibilité par la mise en place de mesures de résilience informatique, réduisent drastiquement l’effet de la cyber attaque américaine. Loin de les effrayer, ils libèrent à leur tour des cyber attaques sur des cibles civiles américaines. Attaques qui génèrent nettement plus de dégâts et un camouflet pour la superpuissance américaine. 

Ainsi, au lieu d’empêcher une escalade militaire, les capacités offensives américaines, en l’absence d’une base défensive forte, ont ici eu l’effet inverse. 

Les auteurs ont ainsi à cœur de démontrer que la volonté de puissance et de domination du cyber espace par les États-Unis ne peut se faire de manière unilatérale et doit passer par un sursaut dans le domaine défensif. Un sursaut qui en 2022, tarde encore à se réaliser outre-Atlantique.
